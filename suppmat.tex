% Options for packages loaded elsewhere
\PassOptionsToPackage{unicode}{hyperref}
\PassOptionsToPackage{hyphens}{url}
%
\documentclass[
  12pt,
]{article}
\usepackage[]{times}
\usepackage{amsmath}
\usepackage{ifxetex,ifluatex}
\ifnum 0\ifxetex 1\fi\ifluatex 1\fi=0 % if pdftex
  \usepackage[T1]{fontenc}
  \usepackage[utf8]{inputenc}
  \usepackage{textcomp} % provide euro and other symbols
  \usepackage{amssymb}
\else % if luatex or xetex
  \usepackage{unicode-math}
  \defaultfontfeatures{Scale=MatchLowercase}
  \defaultfontfeatures[\rmfamily]{Ligatures=TeX,Scale=1}
\fi
% Use upquote if available, for straight quotes in verbatim environments
\IfFileExists{upquote.sty}{\usepackage{upquote}}{}
\IfFileExists{microtype.sty}{% use microtype if available
  \usepackage[]{microtype}
  \UseMicrotypeSet[protrusion]{basicmath} % disable protrusion for tt fonts
}{}
\makeatletter
\@ifundefined{KOMAClassName}{% if non-KOMA class
  \IfFileExists{parskip.sty}{%
    \usepackage{parskip}
  }{% else
    \setlength{\parindent}{0pt}
    \setlength{\parskip}{6pt plus 2pt minus 1pt}}
}{% if KOMA class
  \KOMAoptions{parskip=half}}
\makeatother
\usepackage{xcolor}
\IfFileExists{xurl.sty}{\usepackage{xurl}}{} % add URL line breaks if available
\IfFileExists{bookmark.sty}{\usepackage{bookmark}}{\usepackage{hyperref}}
\hypersetup{
  hidelinks,
  pdfcreator={LaTeX via pandoc}}
\urlstyle{same} % disable monospaced font for URLs
\usepackage[left=2.54cm,right=2.54cm,top=2.54cm,bottom=2.54cm]{geometry}
\usepackage{graphicx}
\makeatletter
\def\maxwidth{\ifdim\Gin@nat@width>\linewidth\linewidth\else\Gin@nat@width\fi}
\def\maxheight{\ifdim\Gin@nat@height>\textheight\textheight\else\Gin@nat@height\fi}
\makeatother
% Scale images if necessary, so that they will not overflow the page
% margins by default, and it is still possible to overwrite the defaults
% using explicit options in \includegraphics[width, height, ...]{}
\setkeys{Gin}{width=\maxwidth,height=\maxheight,keepaspectratio}
% Set default figure placement to htbp
\makeatletter
\def\fps@figure{htbp}
\makeatother
\setlength{\emergencystretch}{3em} % prevent overfull lines
\providecommand{\tightlist}{%
  \setlength{\itemsep}{0pt}\setlength{\parskip}{0pt}}
\setcounter{secnumdepth}{-\maxdimen} % remove section numbering
\usepackage{times}
\renewcommand{\thefigure}{S\arabic{figure}}
\ifluatex
  \usepackage{selnolig}  % disable illegal ligatures
\fi

\author{}
\date{}

\begin{document}

\hypertarget{supplementary-materials}{%
\section{Supplementary materials}\label{supplementary-materials}}

\begin{figure}
\hypertarget{fig:kin_kout_diff}{%
\centering
\includegraphics{figures/kin_kout_difference.png}
\caption{Difference between predicted and empirical values for the
relative number of predators \(k_{in}\) and the relative number of prey
\(k_{out}\) when species are ordered according to (a) their out-degree
and (b) their in-degree. Empirical networks include most food webs
archived on Mangal, as well as the New Zealand and Tuesday lake datasets
(our complete dataset). The predicted joint degree sequences were
obtained after sampling one realization of the joint degree distribution
of maximum entropy for each network, while keeping the total number of
interactions constant. In each panel, each dot corresponds to a single
species in one of the networks.}\label{fig:kin_kout_diff}
}
\end{figure}

\begin{figure}
\hypertarget{fig:heatmap}{%
\centering
\includegraphics{figures/heatmap_disconnected.png}
\caption{Probability that a species is isolated in its food web
according to the degree distribution of maximum entropy. We derived
degree distributions of maximum entropy given a range of values of \(S\)
and \(L\), and plotted the probability that a species has a degree \(k\)
of \(0\) (log-scale color bar). Here species richness varies between
\(5\) and \(100\) species, by increment of \(5\) species. For each level
of species richness, the numbers of links correspond to all 20-quantiles
of the interval between \(0\) and \(S^2\). The black line marks the
\(S-1\) minimum numbers of links required to have no isolated
species.}\label{fig:heatmap}
}
\end{figure}

\begin{figure}
\hypertarget{fig:entropy_dist}{%
\centering
\includegraphics{figures/entropy_distribution.png}
\caption{(a) Distribution of the SVD entropy of MaxEnt food webs (type
II MaxEnt network model) and of empirical food webs. Empirical networks
include most food webs archived on Mangal, as well as the New Zealand
and Tuesday lake datasets (our complete dataset). Maximum entropy
networks were derived using a simulating annealing algorithm to find the
network of maximum SVD entropy while maintaining the joint degree
sequence. (b) Distribution of z-scores of the SVD entropy of all
empirical food webs. Z-scores were computed using the mean and standard
deviation of the distribution of SVD entropy of MaxEnt food webs (type
II MaxEnt network model). The dash line corresponds to the median
z-score.}\label{fig:entropy_dist}
}
\end{figure}

\begin{figure}
\hypertarget{fig:entropy_size}{%
\centering
\includegraphics{figures/difference_entropy.png}
\caption{Difference in SVD entropy between MaxEnt (type II MaxEnt
network model) and empirical food webs as a function of (a) species
richness, (b) the number of links, and (c) connectance. Empirical
networks include most food webs archived on Mangal, as well as the New
Zealand and Tuesday lake datasets (our complete dataset). Maximum
entropy networks were derived using a simulating annealing algorithm to
find the network of maximum SVD entropy while maintaining the joint
degree sequence. Regression lines are plotted in each
panel.}\label{fig:entropy_size}
}
\end{figure}

\begin{figure}
\hypertarget{fig:measures_richness}{%
\centering
\includegraphics{figures/measures_richness.png}
\caption{Structure of empirical and maximum entropy food webs (type II
MaxEnt network model) as a function of species richness. Empirical
networks include most food webs archived on Mangal, as well as the New
Zealand and Tuesday lake datasets (our complete dataset). Maximum
entropy networks were derived using a simulating annealing algorithm to
find the network of maximum SVD entropy while maintaining the joint
degree sequence. (a) Nestedness (estimated with the spectral radius of
the adjacency matrix), (b) the maximum trophic level, (c) the network
diameter (i.e., the longest shortest path between all species pairs),
and (d) the SVD entropy were measured on these empirical and predicted
food webs and plotted against species richness. Regression lines are
plotted in each panel.}\label{fig:measures_richness}
}
\end{figure}

\end{document}
